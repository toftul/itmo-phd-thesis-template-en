\chapter*{Introduction}                         % Заголовок
\addcontentsline{toc}{chapter}{Introduction}    % Добавляем его в оглавление


\paragraph*{Relevance of the topic.}
\paragraph*{Goal of research.}
\paragraph*{Research objectives.}
\paragraph*{The novelty of research.}
\paragraph*{Theoretical and practical meaning of research.}
\paragraph*{Assertions presented for defense.}
\begin{enumerate}
    \item \statementOneEN
    \item \statementTwoEN
\end{enumerate}

\paragraph*{Approbation.}
\paragraph*{Accuracy of obtained results.}
\paragraph*{Implementation of research results.}
\paragraph*{Publications.}
Relevant publications of the author:
\begin{refsection}[biblio/own.bib]
\nocite{*}
\printbibliography[
    keyword=own,
    %title={Список всех публикаций автора по теме диссертации}, 
    %heading=subbibliography,
    heading=none,
    resetnumbers=true
]
\end{refsection}

\paragraph*{Thesis structure and number of pages}
Thesis consists of the introduction,
\formbytotal{totalchapter}{chapter}{}{s}{},
conclusion and 
\formbytotal{totalappendix}{appendix}{}{es}{}.
Thesis is 
\formbytotal{TotPages}{page}{}{s}{} long, including
\formbytotal{totalcount@figure}{figure}{}{s}{} and
\formbytotal{totalcount@table}{table}{}{s}{}.
Bibliography consists of
\formbytotal{citenum}{item}{}{s}{}.



\begin{figure}
    \centering
    \includegraphics[width=0.6\linewidth]{images/knuth}
    \caption{Knuth}
    \label{fig:my_label}
\end{figure}